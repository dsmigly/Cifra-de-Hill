\documentclass[oneside,a4paper,12pt]{article}
\usepackage[portuguese]{babel}
%\usepackage{cat}
\usepackage{estilosa}
\usepackage[alf]{abntex2cite}
%\usepackage[utf8]{inputenc}
\usepackage[T4, T1]{fontenc}
\usepackage{bbm}
\usepackage{ wasysym }
%\usepackage[top=30mm, bottom=30mm, left=30mm, right=30mm]{geometry}
%\usepackage{framed}
%\usepackage{booktabs}
%\usepackage{color}
%\usepackage{hyperref}
%\usepackage{graphicx}
%\usepackage{float}
%\usepackage[table,xcdraw]{xcolor}
%\usepackage{pstricks}
%\usepackage{ mathrsfs }
%\graphicspath{{./Figuras/}}    
%\definecolor{shadecolor}{rgb}{0.8,0.8,0.8}
\newcommand{\AS}[1]{{\fontencoding{T4}\selectfont#1}}


\usepackage{stackengine,graphicx}
\stackMath
\newcommand\frightarrow{\scalebox{1}[.4]{$\longrightarrow$}}
\newcommand\darrow[1][]{\mathrel{\stackon[1pt]{\stackanchor[1pt]{\frightarrow}{\frightarrow}}{\scriptstyle#1}}}
\newcommand\farrow[1][]{\mathrel{\stackon[1pt]{\frightarrow}{\scriptstyle#1}}}
\usepackage{cancel}
\usepackage{tikz}
\usepackage[utf8]{inputenc}
\usepackage{mathtext}
\usepackage{graphicx}
\usepackage[normalem]{ulem}
\usepackage{wrapfig}
\usepackage[T1]{fontenc}
\usepackage{blindtext}
\usepackage{textcomp}
\usepackage{tasks}
\usepackage{amsthm}
\usepackage{setspace}
%\usepackage{polyglossia}
\usepackage{fancybox}
\usepackage{amsmath}
\usepackage{amsfonts}
\usepackage{amssymb}
\usepackage{graphicx, color}
%\newcommand{\sen}{{\rm sen}}
%\newcommand{\tg}{{\rm tg }}
%\newcommand{\cotg}{{\rm cotg }}
%\newcommand{\cossec}{{\rm cossec }}
%\newcommand{\arctg}{{\rm arctg }}
%\newcommand{\arcsen}{{\rm arcsen }}
\newcommand{\negrito}[1]{\mbox{\boldmath{$#1$}}} 
\newtheorem*{axioma}{Axioma da Infinidade}
\usepackage{pifont}
\usepackage[framemethod=TikZ]{mdframed}
%\newcommand{\heart}{\ensuremath\heartsuit}
%\newcommand{\diamonde}{\ensuremath\diamondsuit}
\DeclareSymbolFont{extraup}{U}{zavm}{m}{n}
\DeclareMathSymbol{\varheart}{\mathalpha}{extraup}{86}
\DeclareMathSymbol{\vardiamond}{\mathalpha}{extraup}{87}
\theoremstyle{definition}
%\newtheorem{definicao}{Definição}
%\newtheorem{exemplo}{Exemplo}
%\newtheorem{proposicao}{Exemplo}
%\newtheorem{teorema}{Teorema}
%\newtheorem{lema}{Lema}
%\newtheorem{corolario}{Corolário}
\theoremstyle{plain}
\newtheorem*{afi}{Afirmação}
\newtheorem*{propo}{Proposição}
\newtheorem{dem}{Demonstração}
\newtheorem{teo}{Teorema}
\newtheorem*{lem}{Lema}
\newtheorem{coro}{Corolário}
\newtheorem{afirmacao}{Afirmação}
\newtheorem{exercicio}{Exercício}

%\theoremstyle{Colorido}
\newtheorem{questao}{\textcolor{Floresta}{\textit{\bf Questão}}}
\newtheoremstyle{Colorido}{}{}{\color{Floresta}}{}{\color{Floresta}\bfseries}{}{ }{}
\newtheorem{theorem}{Theorem}
\newtheoremstyle{solu}{}{}{}{}{\color{red}\bfseries}{}{ }{}
\theoremstyle{solu}
\newtheorem*{resp}{Solução}
\newtheoremstyle{dotlessP}{}{}{}{}{\color{Floresta}\bfseries}{}{ }{}
\theoremstyle{dotlessP}
\newtheorem{sol}{Questão}

\newcommand{\solucao}[1]{\textcolor{blue}{\textbf{Solução:} #1}}
\newcommand{\blue}[1]{\textcolor{blue}{#1}}


\newcommand{\definiendum}[2]{\textbf{#1} ALkn \textit{#2}}

%\definiendum{dfd}{hcervh}

\newcommand         \ie     {\textit{i.e.}}


%FAZ EDICOES AQUI (somente no conteudo que esta entre entre as ultimas  chaves de cada linha!!!)
\newcommand{\universidade}{Faculdade de Tecnologia Termomecanica}
\newcommand{\centro}{Centro Educacional da Fundação Salvador Arena}
%\newcommand{\departamento}{Departamento}
%\newcommand{\curso}{Curso}
%\DeclareMathOperator{\Tr}{Tr} \DeclareMathOperator{\Spec}{Spec}
\newcommand{\professor}{Bruno Henrique Michelin Silva, Douglas de Araujo Smigly, Gileade Lamede Martins}
\newcommand{\disciplina}{Tópicos Avançados em Redes}
\newcommand{\nusp}{081160008, 081160030, 081160044}
\newcommand{\sigla}{MAT6705}
\newcommand{\entrega}{9 de abril de 2018}
\DeclareSymbolFont{extraup}{U}{zavm}{m}{n}
\DeclareMathSymbol{\varheart}{\mathalpha}{extraup}{86}
\DeclareMathSymbol{\vardiamond}{\mathalpha}{extraup}{87}
	\cornersize{.3} 
	\mdfdefinestyle{MyFrame}{%
    linecolor=blue,
    outerlinewidth=2pt,
    roundcorner=20pt,
    innertopmargin=\baselineskip,
    innerbottommargin=\baselineskip,
    innerrightmargin=20pt,
    innerleftmargin=20pt,
    backgroundcolor=gray!24!white}
    
  		\cornersize{.3} 
	\mdfdefinestyle{MyFramew}{%
    linecolor=Floresta,
    outerlinewidth=2pt,
    roundcorner=20pt,
    innertopmargin=\baselineskip,
    innerbottommargin=\baselineskip,
    innerrightmargin=20pt,
    innerleftmargin=20pt,
    backgroundcolor=Floresta!24!white}
%ATE AQUI !!!
%ATE AQUI !!!


\makeindex

\begin{document}
\definecolor{Floresta}{rgb}{0.13,0.54,0.13}
	\thispagestyle{empty}
	
	\begin{center}
%	\includegraphics[width=\linewidth/3]{logo_pic}%LOGOTIPO DA INSTITUICAO
	 	\vspace{0pt}
	 	
		\universidade
		\par
		\centro
		\par
%		\departamento
		\par
%		\curso
		\par
		\vspace{24pt}
		\LARGE \textbf{}
		\par
		\par
		\par
		\par
		\LARGE \textbf{Cifra de Hill}
		\par
		\par
		\par
		\par
		\par
	
	\end{center}
	
	\vspace{24pt}
	
	%
%	\begin{tabular}{ |l|p{12cm}| }
%		
%		\hline
%		\multicolumn{2}{|c|}{\textbf{Dados de Identificação}} \\
%			\hline
%		Disciplina:        &    \disciplina          \\
%		\hline
%		Professor:         &    \professor           \\
%	\hline
%	Aluno(a):         &\\
%		\hline
%	Multiplicidade:  & \ \ \ \ \ \ \vline Nível: \vline\\
%	
%		\hline
	%\end{tabular}
	%
	\begin{tabular}{ |l|p{11cm}| }
		
		\hline
		\multicolumn{2}{|c|}{\textbf{Dados de Identificação}} \\
			\hline
		Nome:        &  \professor \\
		\hline
		Número USP:      & \nusp  \\
		\hline
		Curso:        &  \disciplina \\
		\hline
		 Código:        &  \sigla \\
%		\hline
%				Data de entrega:      &  \entrega \\
		\hline
	\end{tabular}
	\vspace{24pt}

	
	%\begin{snugshade}
	%	\section{O... aumento  }  
	%\end{snugshade}

	
\title{Continuity and Uniform Continuity}
\author{521}
\newpage
\tableofcontents

%\begin{center}
% \begin{figure}[!htb]
%    \centering
%    \includegraphics[scale=0.25]{ordinal}
%  \end{figure}
%  \end{center}
%\chapter{dd}

\newpage

\section{Criptoanálise}

Decifre o criptograma 
\begin{center}
\textbf{S \k{T} \~{O} V \^{s} \AS{\O} O \AS{\m{k}} ó M \k{T} \"{I} E \H{U} á}
\end{center}
%S Ţ Õ V ŝ Ø O ƙ ó M Ţ Ï E Ű á

com a chave de criptografia
\[
\begin{pmatrix}
1 & 0 & 0 \\
1 & 3 & 1 \\
1 & 2 & 0
\end{pmatrix}
\]

Seja $M = \begin{pmatrix}
1 & 0 & 0 \\
1 & 3 & 1 \\
1 & 2 & 0
\end{pmatrix}.$ 

Para o processo, assumiremos caracteres Unicode, num intervalo de 0 a 1114110. Assim, teremos $1114111$ caracteres disponíveis, e trabalharemos com matrizes em $\mathcal{M}_{m \times n}(\mathbb{Z}_{1114111}).$ Como $M \in \mathcal{M}_3(\mathbb{Z}_{1114111}),$ devemos organizar as letras do criptograma numa matriz $5 \times 3$ com entradas em $\mathbb{Z}_{1114111}$. Para isso, precisamos do código Unicode de cada caractere:
\begin{center}
\begin{tabular}{|c|c|}
\hline
Letra   & Código \\ \hline
    S   &  83 \\ \hline
\k{T}   &  354\\ \hline
\~{O}   &  213 \\ \hline
V       &  86 \\ \hline
\^{s}   &  349\\ \hline
\AS{\O} &  216\\ \hline
O       &  79    \\ \hline
\AS{\m{k}} & 409    \\ \hline
ó       & 243 \\ \hline
M       & 77  \\ \hline
\k{T}   & 354  \\ \hline
\"{I}   &  207 \\ \hline
E       &  69  \\ \hline
\H{U}   &  368  \\ \hline
á       &   225 \\ \hline
\end{tabular}
\end{center}
E assim, obtemos a seguinte matriz $C \in \mathcal{M}_3(\mathbb{Z}_{1114111}):$
\[
\begin{pmatrix}
83 & 354 & 213  \\ 
86 & 349 & 216  \\ 
79 & 409 & 243  \\
77 & 354  &  207 \\
69 & 368 & 225
\end{pmatrix}
\]
Agora, vamos decompor $C$ em 5 vetores-linha, de modo que
\[
C = \begin{pmatrix}
c_1 \\
c_2 \\
c_3 \\
c_4 \\
c_5
\end{pmatrix}.\]
A matriz $D \in \mathcal{M}_{5 \times 3}(\mathbb{Z}_{1114111})$ que representam os códigos decimais dos caracteres da palavra decriptografada será da forma 
\[
D = \begin{pmatrix}
d_1 \\
d_2 \\
d_3 \\
d_4 \\
d_5
\end{pmatrix},\]
onde 
\[
d_i = (M^{-1} c_i)^T, \quad \forall i \in \{1, 2, 3, 4, 5 \}.
\]

Para realizar este procedimento, precisamos garantir que $M$ possui inversa em $\mathbb{Z}_{1114111}.$ Pela Regra de Laplace, 
\[
\det \begin{pmatrix}
1 & 0 & 0 \\
1 & 3 & 1 \\
1 & 2 & 0
\end{pmatrix} = 1 \left| \begin{array}{cc} 3 & 1\\ 2 & 0  \end{array} \right| = -2.
\]
Como $-2$ é invertível em $\mathbb{Z}_{1114111}$ (de fato, $1114111 = 17 \cdot 2^{16} - 1$ é um número primo, e sabemos que $\mathbb{Z}_p$ é um corpo para todo $p$ primo, então todo elemento não-nulo é invertível), segue que a matriz $M$ é invertível em $\mathbb{Z}_{1114111}.$ Para computá-la, utilizaremos a seguinte fórmula:
\[
M^{-1} = (\det{M})^{-1} \mathrm{Adj}(M),
\]
onde $\mathrm{Adj}(M)$ representa a matriz adjunta de $M,$ definida como a matriz transposta da matriz de cofatores. Em nosso caso, temos que $\det(M)^{-1} = (-2)^{-1} = 557055,$ pois $(-2) \cdot 557055 \equiv 1 \pmod{1114111}.$ A matriz adjunta será
\[
\mathrm{Adj}(M) = (\mathrm{cof}(M))^T = \begin{bmatrix}
+\left| \begin{matrix} 3 & 1 \\ 2 & 0 \end{matrix} \right| &
-\left| \begin{matrix} 1 & 1 \\ 1 & 0 \end{matrix} \right| &
+\left| \begin{matrix} 1 & 3 \\ 1 & 2 \end{matrix} \right| \\
 & & \\
-\left| \begin{matrix} 0 & 0 \\ 2 & 0 \end{matrix} \right| &
+\left| \begin{matrix} 1 & 0 \\ 1 & 0 \end{matrix} \right| &
-\left| \begin{matrix} 1 & 0 \\ 1 & 2 \end{matrix} \right| \\
 & & \\
+\left| \begin{matrix} 0 & 0 \\ 3 & 1 \end{matrix} \right| &
-\left| \begin{matrix} 1 & 0 \\ 1 & 1 \end{matrix} \right| &
+\left| \begin{matrix} 1 & 0 \\ 1 & 3 \end{matrix} \right|
\end{bmatrix}^T = \begin{pmatrix}
-2 & 0 & 0 \\
1 & 0 & -1 \\
-1 & -2 & 3\\
\end{pmatrix}.
\]
Logo,
\[
M^{-1} = 557055 \begin{pmatrix}
-2 & 0 & 0 \\
1 & 0 & -1 \\
-1 & -2 & 3\\
\end{pmatrix} = \begin{pmatrix}
1 & 0 & 0 \\
557055 & 0 & 557056 \\
557056 & 1 & 557054\\
\end{pmatrix}
\]
Agora, estamos aptos a determinar $d_i,$ para $i = 1, \ldots, 5.$ Lembrando que 
\[ C = \begin{pmatrix}
83 & 354 & 213  \\ 
86 & 349 & 216  \\ 
79 & 409 & 243  \\
77 & 354 & 207 \\
69 & 368 & 225
\end{pmatrix} = \begin{pmatrix}
c_1 \\
c_2 \\
c_3 \\
c_4 \\
c_5
\end{pmatrix} \Rightarrow \begin{cases}
c_1 = (83, 354, 213) \\
c_2 = (86, 349, 216) \\
c_3 = (79, 409, 243) \\
c_4 = (77, 354, 207) \\
c_5 = (69, 368, 225)
\end{cases}.
\]
Com essas informações em mãos, e lembrando que os cálculos estão sendo realizados em $\mathbb{Z}_{1114111},$ temos
\[
d_1 = (M^{-1}c_1)^T =  \left(\begin{pmatrix}
1 & 0 & 0 \\
557055 & 0 & 557056 \\
557056 & 1 & 557054\\
\end{pmatrix}  \begin{pmatrix}
83 \\
354 \\
213 \\
\end{pmatrix}\right)^T = \left(\begin{matrix}
83 \\
164888493 \\
164888504
\end{matrix}\right)^T = (83, 65, 76)
\]
\[
d_2 = (M^{-1}c_2)^T =  \left(\begin{pmatrix}
1 & 0 & 0 \\
557055 & 0 & 557056 \\
557056 & 1 & 557054\\
\end{pmatrix}  \begin{pmatrix}
86 \\
349\\
216 \\
\end{pmatrix}\right)^T = \left(\begin{matrix}
86 \\
168230826 \\
168230829
\end{matrix}\right)^T = (86, 65, 68)
\]
\[
d_3 = (M^{-1}c_3)^T =  \left(\begin{pmatrix}
1 & 0 & 0 \\
557055 & 0 & 557056 \\
557056 & 1 & 557054\\
\end{pmatrix}  \begin{pmatrix}
79 \\
409\\
243 \\
\end{pmatrix}\right)^T = \left(\begin{matrix}
79 \\
179371953 \\
179371955
\end{matrix}\right)^T = (79, 82, 84)
\]
\[
d_4 = (M^{-1}c_4)^T =  \left(\begin{pmatrix}
1 & 0 & 0 \\
557055 & 0 & 557056 \\
557056 & 1 & 557054\\
\end{pmatrix}  \begin{pmatrix}
77 \\
354\\
207 \\
\end{pmatrix}\right)^T = \left(\begin{matrix}
77 \\
158203827 \\
158203844
\end{matrix}\right)^T = (77, 65, 82)
\]
\[
d_5 = (M^{-1}c_5)^T =  \left(\begin{pmatrix}
1 & 0 & 0 \\
557055 & 0 & 557056 \\
557056 & 1 & 557054\\
\end{pmatrix}  \begin{pmatrix}
69 \\
368\\
225 \\
\end{pmatrix}\right)^T = \left(\begin{matrix}
69 \\
163774395 \\
163774382
\end{matrix}\right)^T = (69, 78, 65)
\]
Portanto, a matriz $D \in \mathcal{M}_{5 \times 3}(\mathbb{Z}_{1114111})$ que contém os valores decimais do criptograma será
\[
D = \begin{pmatrix}
d_1 \\
d_2 \\
d_3 \\
d_4 \\
d_5
\end{pmatrix} = \begin{pmatrix}
83 & 65 & 76 \\ 
86 & 65 & 68 \\ 
79 & 82 & 84 \\  
77 & 65 & 82 \\ 
69 & 78 & 65 
\end{pmatrix}
\]

Agora, basta encontrar os respectivos caracteres que correspondem aos códigos Unicode acima descritos:
\begin{center}
    \begin{tabular}{|c|c|}
    \hline
    Código & Letra \\ \hline
    65     &  A \\ \hline
    68     &  D \\ \hline
    69     &  E \\ \hline
    76     &  L \\ \hline
    77     &  M \\ \hline
    78     &  N \\ \hline
    79     &  O \\ \hline
    82     &  R \\ \hline
    83     &  S \\ \hline
    84     &  T \\ \hline
    86     &  V \\ \hline
    \end{tabular}
\end{center}

Portanto, 
\begin{center}
    \begin{tabular}{|c|c|c|c|c|c|c|c|c|c|c|c|c|c|c|}
    \hline
      83 & 65 & 76 & 86 & 65 & 68 & 79 & 82 & 84 & 77 & 65 & 82 & 69 & 78 & 65 \\ \hline
      S & A & L & V & A & D & O & R & T & M & A & R & E & N & A \\ \hline
    \end{tabular}
\end{center}

Assim, a mensagem decriptografada pela cifra de Hill é \newline 
\newline \centering \textbf{SALVADORTMARENA}.
\end{document}